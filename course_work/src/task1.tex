\section{Патентный поиск}
\label{sec:patent}
Для поиска аналогов разрабатываемого устройства воспользуемся интернет-ресурсом Федерального института промышленной собственности (\href{https://www1.fips.ru/iiss/db.xhtml}{https://www1.fips.ru/iiss/db.xhtml}) и рассмотрим различные в реализации варианты решения аналогичных задач.
\subsection{Измерение на основе понетциометра}

Принцип работы первого аналога рассматривается в описании изобретения к патенту \textbf{ИЗ №2780031} (Приложение А).

Устройство можно разделить на 3 основные части: источник опорного напряжения, потенциометрический датчик и микроконтроллер для считывания напряжения с датчика.

При измерении используется связь между контролируемым объектом и подвижным контактом потенциометрического датчика.
Датчик закрепляется непосредственно на вращающуюся ось подвижной частью, в неподвижной -- к неподвижной опоре вращающейся оси.
Величина измеряемого напряжения на датчике связана с физическим положением объекта.

Преимуществами такого исполнения измерительного устройства являются:
\begin{itemize}
    \item Относительная простота реализации: датчик представляет из себя переменный делитель напряжения, не требующий сложных вычислений при преобразовании сигнала.
    \item Энергонезависимость датчика: после отключения датчика от источника питания и повторного включения показания не изменяться, в отличие от датчиков, подобных относительным энкодерам.
    \item Компактность измерительного устройства: современный ассортимент потенциометров предлагает весьма широкий выбор устройств с различными параметрами и габаритами.
\end{itemize}

К недостаткам можно отнести:
\begin{itemize}
    \item Необходимость создания источника опорного напряжения.
    \item Стирание резистивного слоя внутри датчика, что ведет к ухудшению точности измерений.
\end{itemize}

\subsection{Измерение на основе магниточувствительного элемента}

Принцип работы второго аналога рассматривается в описании полезной модели к патенту \textbf{ПМ №175216} (Приложение Б).

Полезная модель, описанная в работе, относится к измерительной технике и может быть использована для бесконтактных абсолютных измерений угловых перемещений.
Устройство для измерения угловых перемещений включает подвижный модуль, выполненный в виде вала с винтовой канавкой, образующей на поверхности вала чередующиеся выступы и впадины, сопряженный с объектом пользователя, и магнитный преобразователь, дистанционно взаимодействующий с подвижным модулем, содержащий магниточувствительные элементы, сопряженные с подвижным модулем и плату обработки.

Преимуществами такого исполнения измерительного устройства являются:
\begin{itemize}
    \item Бесконтактный съем показаний: наличие датчика не влияет на свободу перемещения подвижного объекта.
            Так же отсутствие физического контакта повышает износостойкость измерительного прибора.
    \item Малые габариты: современные магниточувствительные элементы имеют малые габариты, не теряя при этом точности измерений.
\end{itemize}

К недостаткам можно отнести:
\begin{itemize}
    \item Необходимость установки резьбового вала на ось вращения зеркала.
    \item Необходимость использования как минимум двух чувствительных элементов для определения направления вращения.
    \item Необходимость установки концевика для калибровки или ручной калибровки положения после каждого отключения от питания.
\end{itemize}

\subsection{Измерение на основе кодового лимба}
Принцип работы третьего аналога рассматривается в описании изобретения к патенту \textbf{ИЗ №2120105} (Приложение В).

Преобразователь угловых перемещений содержит подвижную систему, состоящую из жестко закрепленного на валу кодового лимба, и систему считывания информации, состоящую из последовательно расположенных источников излучения, конденсора и фотоприемного устройства.
Преобразователь содержит также электронный блок в виде блока аналого-цифровой обработки и управления, вход которого связан с выходами фотоприемного устройства.
На кодовый лимб преобразователя наносится одна кодовая дорожка со штрихами.

Преимущества данного способа:
\begin{itemize}
    \item Бесконтактный съем показаний
    \item Энергонезависимость датчика при нанесении уникальной кодировки
\end{itemize}

Недостатки:
\begin{itemize}
    \item Дискретность сигнала, шаг дискретности зависит от физических способностей фотоприемного устройства и возможности точного изготовления кодового лимба.
    \item Необходимость точного исполнения кодового лимба. Небольшие отклонения могут сильно сказаться на точности датчика.
\end{itemize}

\subsection{Выбор принципа действия разрабатываемого устройства}\label{subsec:choise1}

Опираясь на анализ преимуществ и недостатков каждого из способов, было принято решение проводить дальнейшую разработку, опираясь на способ с использованием \underline{потенциометрического} датчика.

Следующие этапы разработки будут включать поиск информации о потенциометрических датчиках, разработку функциональной схемы устройства, разработку принципиальной электрической схемы и выбор источников питания.
Проведенный поиск аналогов будет служить одним из источников информации при их выполнении.
