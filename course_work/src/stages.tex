\section{Этапы выполнения}
\label{sec:stages}
\begin{table}[!h]
    \centering
    \begin{tabularx}{\textwidth}{
        | >{\raggedright\arraybackslash}>{\hsize=.27\hsize}X
        | >{\centering\arraybackslash}>{\hsize=.45\hsize}X
        | >{\raggedright\arraybackslash}X
        | >{\centering\arraybackslash}>{\hsize=.3\hsize}X
        | >{\centering\arraybackslash}>{\hsize=.9\hsize}X|}
        \hline
        \textbf{Номер этапа} & \textbf{Название этапа} & \textbf{Содержание} & \textbf{Срок исполнения} & \textbf{Дополнительное задание} \\
        \hline \hline
        1 & Патентный поиск & 1. Поиск аналогов по источникам патентной информации (3 аналога разного принципа действия). \newline 2. Сравнительный анализ выбранных аналогов. \newline 3. Выбор принципа действия разрабатываемого устройства. & до 15.10.22 & Поиск двух зарубежных аналогов с включением их в сравнительный анализ \\
        \hline
        2 & Техническое предложение & 1. Библиографический поиск информации о датчиках, выбранного принципа действия (расчетные формулы, схемы включения и т.д.) \newline 2. Разработка функциональной схемы устройства. & до 15.11.22 & Чертеж функциональной схемы \\
        \hline
        3 & Разработка собственного технического решения & 1. Разработка принципиальной электрической схемы (или схемы соединений) вторичного преобразователя (наличие микроконтроллера в схеме – обязательно) \newline 2. Выбор источника(ов) питания. \newline 3. Чертеж Э3 или Э4 & до 30.12.22 & Разработка алгоритма функционирования микроконтроллера \\
        \hline
    \end{tabularx}
    \caption{Этапы выполнения задания}
    \label{tab:stages}
\end{table}
\newpage
