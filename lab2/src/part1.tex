\section{Марериалы работы}

\subsection{Основные технические характеристики исследуемого датчика}
Исследуемый датчик -- инкрементальный энкодер.
Относится к типу энкодеров, которые предназначены для указания направления движения и/или углового перемещения внешнего механизма.
Инкрементальный энкодер формирует импульсы, количество которых соответствует повороту вала на определенный угол.
Этот тип энкодеров, в отличие от абсолютных, не формирует код положения вала, когда вал находится в покое..

Паспортные характеристики исследуемого энкодера E50S8:
\begin{itemize}
    \item Диаметр корпуса: 50 мм
    \item Диаметр вала: 8 мм
    \item Количество импульсов на оборот вала: 1000
    \item Выходной сигнал: дифференциальный, парафазный
    \item Напряжение питания: 5В постоянного тока
\end{itemize}

\subsection{Экспериментальная установка и змерительные средства}
Помимо прочего, блок содержит приводной двигатель постоянного тока и энкодер, соединенные зубчатой передачей.
\begin{figure}[!h]
    \centering
    \includegraphics[width=0.8\textwidth]{img/Screenshot_20221002_213647}
    \caption{Панель экспериментальной установки}
    \label{fig:Screenshot_20221002_213647}
\end{figure}

Угол поворота энкодера и количество импульсов отображаются на одном (нижнем) индикаторном блоке.
Переключение режимов отображения осуществляется тумблером.

\subsection{Результаты измерений и их обработка}
Зафиксируем статическую характеристику энкодера на холостом ходу.
\begin{table}[!h]
    \centering
    \caption{Статическая характеристика энкодера}
    \label{tab:tabl}
    \begin{tabular}{|c|c|}
        \hline
        Угол поворота $\alpha,^\circ$& Число импульсов $N$ \\\hline
        0&	0\\
        30&	84\\
        60&	169\\
        90&	252\\
        120&	335\\
        150&	417\\
        180&	502\\
        210&	584\\
        240&	669\\
        270&	751\\
        300&	835\\
        330&	919\\
        360&	1000\\
        \hline
    \end{tabular}
\end{table}

Построим график:

\begin{figure}[!h]
    \centering
    \includegraphics[width=0.8\textwidth]{img/task1_r_0}
    \caption{Статическая характеристика энкодера }
    \label{fig:task1_r_0}
\end{figure}

\subsection{Расчет погрешностей}
Найдем максимальное значение абсолютной и относительной погрешностей:
\begin{table}[!h]
    \centering
    \label{tab:otkl}
    \begin{tabular}{|c|c|}
        \hline
        Наибольшая $\Delta N$ & $\epsilon,\%$\\        \hline
        2.33& 1.4\\ \hline
    \end{tabular}
    \caption{Значения погрешностей}

\end{table}

\newpage
\section{Вывод}
В ходе работы мы на практике поработали с инкрементальным энкодером и построена его статическая характеристика.
Было установлено, что статическая характеристика энкодера линейная.